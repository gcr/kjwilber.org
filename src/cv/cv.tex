% thanks to http://texblog.org/2012/04/25/writing-a-cv-in-latex/
\documentclass[10pt,letterpaper]{article}
\usepackage[margin=0.6in]{geometry}
\usepackage{tikz}

%\usepackage{showframe} % Debug margins
\usepackage{graphicx}

\usepackage{amssymb}

\usepackage{enumitem}
\setlist{nosep}

\usepackage{longtable}
\setlength{\LTpre}{0pt}
\setlength{\LTpost}{0pt}

\usepackage[math]{kurier} % nicer fonts
% \usepackage[osf]{mathpazo} % Palatino, which is not bad.
% \usepackage{tgpagella}
\usepackage{url}

\usepackage[T1]{fontenc}
\usepackage{microtype}
\usepackage[us,12hr]{datetime}
\setlength\parindent{0pt}
\usepackage{array, xcolor}
\definecolor{lightgray}{gray}{0.8}
\newcolumntype{L}{>{\raggedleft}p{0.12\textwidth}}
\newcolumntype{R}{p{0.85\textwidth}}
\newcommand\VRule{\color{lightgray}\vrule width 0.5pt}


\usepackage{datetime}
\usepackage{fancyhdr}
\pagestyle{fancyplain}
\fancyhf{}
\rhead{\fancyplain{}{Page \thepage}}
\chead{\fancyplain{Last modified: \today}{Kimberly Wilber}}
\renewcommand{\headrulewidth}{0pt}

\begin{document}
\thispagestyle{plain}
\begin{minipage}{0.60\textwidth}
  \flushleft
{\bfseries\Huge Kimberly J. Wilber}
\vspace{-0.8em}
\rule[0.7em]{\linewidth}{0.5mm}
%\vspace{0.2em}
Software Engineer, Google AI
\hfill {She / Her}
\end{minipage}
\hfill
\begin{minipage}[ht]{0.38\textwidth}
\flushright{%wilmic1102@gmail.com
kimmy@kjwilber.org
\\{http://kjwilber.org}
% \\111 8th Ave, Ste. 302
% \\New York, NY 10011
%\\(719) 237-8834
%\\\url{http://github.com/gcr}
}
\end{minipage}
\section*{Technical Skills}
\vspace{-4pt}
\begin{itemize}
% \item \textbf{Biometrics and template protection.} \\
%   Designed and implemented ``Vaulted Verification,'' a secure
%   client/server authentication protocol. Applied Vaulted Verification
%   to face and iris verification.
% \item \textbf{Face recognition.} \\
%   Helped
\item \textbf{Research interests:} Computer vision, crowdsourcing, and machine learning. Specific areas include biometrics, object recognition, perceptual embeddings, and artistic style recognition.

\item \textbf{Languages and Libraries:}
  Fluent in Python, the scientific Python stack (numpy/scipy,
  Cython, scikit-learn, scikit-image), Javascript, and
  general purpose POSIX tools. %Reasonably familiar with Lua/Torch7,
  Familiar with ML/CV tools (Torch7, Pytorch, OpenCV).
  %Racket/Scheme, node.js,
  %C (and to a lesser extent, C\kern-0.2ex\raisebox{0.4ex}{\scalebox{0.8}{+\kern-0.4ex+}}),
  %\textsc{Matlab}, Java, and C\#.
  %Familiar with several graphics libraries including OpenGL, OpenCV.
  %Also fluent in presentation languages including \LaTeX , HTML, and CSS.
  Contributor to open-source projects including node.js and Racket. Some experience with C and Java.
\item \textbf{Linux Server Administration:}
  Ten years of Debian, Ubuntu, and Arch Linux experience on server,
  desktop, and cloud services (EC2, Digital\-Ocean). Managed over 20
  Debian servers at
  startups and university research labs.
%Vision and Security Technology Laboratory at
% the University of Colorado Colorado Springs (UCCS) and Securics,
% Inc.
%\item \textbf{Cluster Computing and ``Big Data'':}
%  Skilled at parallelizing large experiments. Created internal
%  projects with BOINC, the \emph{Berkeley Open Infrastructure for
%    Network Computing}. Designed several frameworks for distributed
%  computing in Python and
%  Racket. %; subsequently used such frameworks on
%  % 20 machines mentioned above to ensure experiments mentioned above
%  % finished before deadline.
%Can port \textsc{Matlab}/Python code to C.
\end{itemize}
% Languages?
% play up "big data" system design
\vspace{-8pt}

\section*{Education}
\begin{longtable}{L!{\VRule}R}
2014--2018&\textbf{Ph.D. in Computer Science, Cornell Tech}\\
&{\small Supported by the National Science Foundation Graduate Research Fellowship~(NSF~GRFP)}\vspace{5pt}\\
2013--2014&Graduate studies at University of California, San Diego\\
&{\small Transferred to Cornell to follow my advisor, Dr. Serge Belongie}\vspace{5pt}\\
2009--2013&\textbf{Bachelor of Innovation in Computer Science, University
of Colorado Colorado Springs} %(4.0~GPA)
%\\
%&{\small Supported by the Kane Family Foundation Scholarship, Braxton Scholarship}
\vspace{5pt}\\
% &{\small The Bachelor of Innovation is like a BS but places more emphasis on
% teamwork and technical writing.}
2008--2010&High-school concurrent classes at University of Colorado
Colorado Springs %(4.0~GPA)
\vspace{5pt}\\
2007--2008&High-school concurrent classes at Colorado Technical
University %(4.0~GPA)
\end{longtable}
\vspace{5pt}

\section*{Professional Experience}
% REU program
% copy papers from website
\begin{longtable}{L!{\VRule}R}
  2018--Present&\textbf{Software Engineer, Google AI, New York, NY}
\begin{itemize}
\item Developing large-scale machine learning and computer vision applications in
    collaboration with multiple teams.
\vspace{-5pt}
\end{itemize}
\\
  2017&\textbf{Summer Intern, Google Photos Team, Mountain View, CA}
\begin{itemize}
\item Implemented and tested prototyping tools for new user interactions
\vspace{-5pt}
\end{itemize}
\\
  2016&\textbf{Summer Intern, Adobe Research, San Jose, CA}
\begin{itemize}
\item Built systems to analyze millions of images and terabytes of data
\vspace{-5pt}
\end{itemize}
\\
  2014&\textbf{Summer Intern, Dropbox Photos Team, San Francisco, CA}
\begin{itemize}
\item Conducted product-focused computer vision research.
\item Introduced our team to more efficient tools and technologies.
\item Maintained a computer vision evaluation and experimentation pipeline.
\vspace{-5pt}
\end{itemize}
\\
  2014--2018&\textbf{Research Assistant, Cornell University, Cornell Tech NYC}
\begin{itemize}
\item Conducting research related to many areas of computer vision, including perceptual similarity, large-scale crowdsourcing, and object recognition.
\item Helping establish and maintain the new vision group's presence at Cornell.
\item Serving as TA for classes including four semesters of ``CS5785
  Modern Analytics.''
\vspace{-5pt}
\end{itemize}
\\
  2013&\textbf{Research Assistant, University of California, San
    Diego, CA}
\begin{itemize}
\item Conducted computer vision research: face
  recognition, object recognition, perceptual similarity.
\item Helped maintain servers and lab equipment.
\vspace{-5pt}
\end{itemize}
\\
2012--2013&\textbf{Software Engineer, Securics, Inc., Colorado
  Springs, CO}
\begin{itemize}
\item Helped implement ``MugHunt,'' an attribute face search
  engine. MugHunt was one of the most popular demos in its
  session at CVPR 2012.
\item Conducted face recognition experiments to evaluate academic and
  commercial algorithms.
%\item Performed research involving animal recognition in the
  %Mojave desert.
\vspace{-5pt}
\end{itemize}
\\
2009--2013&{\bf Assistant Researcher, Vision and Security Technology
  (VAST) Laboratory at UCCS, CO}
\begin{itemize}
\item Maintained laboratory equipment and over 20 Debian servers.
\item Performed research on face detection and biometrics, including
  biometric template protection.
\item Designed and implemented a cluster computing framework for large-scale fingerprint matching.
\item Helped organize the \emph{Face and Eye Detection on
    Hard Datasets} Competition, IJCB 2011.
\vspace{-5pt}
\end{itemize}
\\
2011&{\bf Summer Researcher, NSF REU Program, University of Colorado
  Colorado Springs, CO}
\begin{itemize}
\item Designed and implemented a privacy-enhanced biometric
  authentication protocol, ``Vaulted Verification.'' This work
  resulted in a provisional patent application, two first-author
  conference papers, and scored fourth place in the \emph{2012 National
    Security Innovation Competition} sponsored by the National
  Homeland Defense Foundation.
% \item Involved researching privacy-protected biometrics, security
%   issues in networks and computer vision, and developing a framework
%   for massively parallel distributed computing to complete experiments
\vspace{-5pt}
\end{itemize}
\\
2009--2010&\textbf{NSF RAHSS High School Intern, Securics, Inc.,
  Colorado Springs, CO}
\begin{itemize}
\item Helped implement ``Verified Presence,'' a time-tracker
  kiosk system that allows employers to verify employees' physical
  attendance with fingerprints.
\item Helped test and debug ``EPayNotary,'' a payment verification
  service that integrates with PayPal. EPayNotary protects customers
  by verifying the identity of merchant recipients.
\vspace{-5pt}
\end{itemize}
\end{longtable}

\section*{Publications}
\begin{longtable}{L!{\VRule}R}
&{\textit{Note that some work before 2018 is published under a previous name.}}\vspace{10pt}\\
<% for paper in papers.paperList %>
<%- if paper.best_paper  %>$\bigstar$ <%  endif -%>
<< paper.year >> & 
<%-  if paper.best_paper -%>
  \textbf{Best paper award: }
<%- endif -%>
\textit{<<paper.title>>}\\
&{\small 
  <% for author in paper.authors -%>
    <%- if "Wilber" in author -%>
      \textbf{<<author>>}
    <%- else -%>
      <<author>>
    <%- endif -%>
    <% if not loop.last %>; <% endif %>
  <%- endfor %>
  \emph{<< paper.venue | replace(" 20", "~20") >>}
}\vspace{10pt}\\
<% endfor %>
\end{longtable}

% \vspace{3pt}
% \section*{Conferences and Scholarships}
% \begin{longtable}{L!{\VRule}R}
% & \textbf{Student Volunteer} at several conferences:\\
% CVPR & IEEE Conference on Computer Vision and Pattern Recognition 2011, 2012, 2013, 2014, 2015\\
% WACV & Winter conference on Applied Computer Vision 2012, 2013, 2014
% \vspace{3pt}\\

% 2013--2016&\textbf{National Science Foundation GRFP Awardee}, UCSD/Cornell\vspace{3pt}\\

% 2010--2013&\textbf{Dean's List}, UCCS\vspace{3pt}\\

% 2010--2013&\textbf{Kane Family Foundation Scholarship Recipient},
% Full tuition and books, UCCS\vspace{3pt}\\

% 2010--2013&\textbf{Braxton Scholarship Recipient}, UCCS

% \end{longtable}
% \vspace{3pt}
% FDHD at IJCB2012

\end{document}

% Education:
% Fall 2010 – Present
% Fall 2008 – Fall 2010
% High-school concurrent student at UCCS
% Fall 2007 - Fall 2008
% Student for one year at Colorado Springs Early Colleges. Transferred to UCCS with a 4.0 GPA.
% Relevant Skills:
% Employment
% Works at the UCCS Vision And Security Technology lab from 2009 – present. Worked at Securics, Inc. as a National Science Foundation high school intern from Winter 2009 – Summer 2010.
% Programming
% Fluent in C#, Python,  Javascript, HTML/CSS, and general purpose Debian Linux tools; two formal courses of Java and C; limited experience with C++, PHP, MySQL, Ruby, Racket, and OpenGL.
% Administration
% Six years of Debian and Ubuntu home server administration experience, working for two years as an assistant system administrator for VAST lab, family wireless network / printer manager, 10+ years unpaid technical support for friends and family, general knowledge of Windows operating systems.
% Projects
% Past projects include “Goggles,” a web-based collaborative drawing tool currently licensed to local company Synapse Software, a language-agnostic robot programming game, a web-based collaborative text editor, and other exploratory projects.
% Extracurricular:
% Played piano from 2000-2007, participated in many recitals. Boy Scout from 2004-2007, earned the rank of Star Scout. Participated in the 2006 Wapiti Boy Scout Leadership Training course. Awarded “2005 Justin Morrill Memorial Award” at an Innovations with Industry summer electrical engineering program. Currently learning guitar.
% References are available on request.
